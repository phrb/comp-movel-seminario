\section{Introdução}

\begin{frame}
    \frametitle{Sobre o Artigo}
    \alert{Título:} 3D real-time indoor localization via broadband nonlinear backscatter in
    passive devices with centimeter precision

    \alert{Autores:} Yunfei Ma, Xiaonan Hui Edwin C. Kan, da \alert{Cornell University}

    \alert{Publicado em:} MobiCom '16 Proceedings of the 22nd Annual
    International Conference on Mobile Computing and Networking Pages 216-229

\end{frame}

\begin{frame}
  \frametitle{Contribuições Principais}

    Monitorar a posição de etiquetas RFID no espaço 3D:

    \begin{itemize}
      \item Com \alert{alta precisão} (erro de $3.5$cm)
      \item Em \alert{tempo real}
      \item Funciona em \alert{ambientes fechados e cheios de objetos}
    \end{itemize}

    Hardware \& software co-design:

    \begin{itemize}
        \item Combinação de frequências
        \item Modificações no algoritmo HMFCW
        \item Algoritmo de Localização 3D
        \item Tag RFID não-linear
    \end{itemize}
\end{frame}

\subsection{Aplicações}

\plain{Aplicações}

\begin{frame}
  \frametitle{Interação Humano-Máquina}
  \begin{itemize}
    \item  Controle/monitoração de braços robóticos
  \end{itemize}

  \begin{center}
    \includegraphics[width=.8\textwidth]{robot-arm}
  \end{center}
\end{frame}

\begin{frame}
  \frametitle{Interação Humano-Máquina}
  \begin{itemize}
    \item  Carros autônomos
  \end{itemize}

  \begin{center}
    \includegraphics[width=\textwidth]{autopilot}
  \end{center}
\end{frame}

\begin{frame}
\frametitle{Jogos \& Realidade aumentada}
  \begin{itemize}
    \item  Kinect
    \item  Wii
  \end{itemize}
\end{frame}

\begin{frame}
  \frametitle{Aplicações Comerciais}
  \begin{itemize}
    \item  Amazon store
    \item  Controle de estoque
  \end{itemize}
\end{frame}

\begin{frame}
  \frametitle{3D-Tracking Hoje}

  Boa parte das técnicas de 3D-Tracking atuais são baseadas em visão
  computacional, com pequenas variações no:

  \begin{itemize}
    \item Aúmero de câmeras utilizadas
    \item Algoritmos de CV utilizados
    \item $\dots$
  \end{itemize}
\end{frame}

\begin{frame}
  \frametitle{Limitações Atuais}

  O problema já não está resolvido?

  \begin{itemize}
    \item  Alta complexidade computacional
    \item  Custo de implementação
    \item  Não funcionam em ambientes com obstáculos
  \end{itemize}
\end{frame}


\begin{frame}
  \frametitle{3D-Tracking usando RFIDs}

  Vantagens:
  \begin{itemize}
    \item  Tags são muito baratas ($0.1 \sim 0.2$ dólares)!
    \item  Técnica funciona mesmo com obstáculos
    \item  Menor complexidade computacional!
  \end{itemize}

  Problemas do estado-da-arte que a técnica do artigo supera:
  \begin{itemize}
    \item  Dependência de antenas de referência ou \emph{anchor nodes}
    \item  Dependência de conhecimento prévio da trajetória
    \item  Limitação a 2D-Tracking
  \end{itemize}
\end{frame}
